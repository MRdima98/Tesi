\documentclass[12pt]{article}
\usepackage{mathptmx}
\usepackage{ragged2e}
\author{}
\title{
    \huge 
        \textbf{Università degli Studi di Modena e Reggio Emilia}
    \large
        \par Dipartimento di Scienze Fisiche, Informatiche e Matematiche
        \par Corso di laurea in Informatica
    \vfil
        \huge \par \textbf{ERS}
    \vfil
}
\date{Anno academico 2018/2019}
\linespread{1.5}
\pagenumbering{arabic}
\begin{document}
\maketitle
\par Introduzione
\par 
\section{Presentazione della situazione esistente}
\subsection{Engim srl}
Engim è un azienda informatica piccola che ha principalmente due prodotti, i
servizio gps e twicetouch. Il primo è il noleggio di tracker gps principalmente 
ai comuni. Questi tracker vengono usati per tracciare la neve pulita e il 
sale sparso. Twicetouch è il noleggio di dispositivi di sicurezza. In particolare 
il badge è un DPI che nel caso in cui il lavoratore cada oppure rimanga fermo 
per un tempo eccessivo, presumibilmente svenuto, il dispostivo manda messaggi 
e effettua chiamate a dei numeri di scelta. 
Il prodotto principale di engim sono i tracker.
\subsection{Infrastuttura}
Engim è organizzata su vari server. Le tecnologie usate sono Ruby on rails full
stack per la parte web e android studio per la parte di app. Onde di evitare 
sovvracarico su un singolo server l'azienda ha molteplici server, dividendo 
quindi il lavoro. Inoltre esiste un sistema di reportistica realizzato direttamente
dentro rails, ma questo ha una serie di problemi:

- La gemma(denominazione delle librerie di ruby) utilizzata per creare i

database non è mantenuta e inoltre 
% \par Requisiti e progetto
% \par Implementazione
% \par Applicazione di esempio
% \par Conclusioni
% \par Appendici
% \par Bibliografia
\end{document}