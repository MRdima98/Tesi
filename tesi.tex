\documentclass[12pt]{article}
\usepackage{amsmath}
\author{}
\title{
    \huge 
        \textbf{Università degli Studi di Modena e Reggio Emilia}
    \large
        \par Dipartimento di Scienze Fisiche, Informatiche e Matematiche
        \par Corso di laurea in Informatica
    \vfil
        \huge \par \textbf{Engim report service}
    \vfil
    \normalsize
    \begin{tabular}{lp{0.5\textwidth}l}
      Relatore: & & Candidato: \\
      Prof Canali & &  Dumitru Frunza \\
      \end{tabular}
}
\date{Anno academico 2021/2022}
\linespread{1.5}
\pagenumbering{arabic}
\begin{document}
\maketitle
\section{Engim Srl}
\subsection{L'azienda}
Engim è una società che si occupa di creare soluzioni tecnologiche in ambito 
ICT, telecomunicazioni, sistemi di gestione e mobilità. Da oltre 10 anni operano 
nel mercato della tracciabilità di flotte e attività e della 
sicurezza dei lavoratori in solitario.
\\ ServizioGPS è il noleggio di tracker gps per veicoli lavoratori. 
Una sostanziosa parte dei clienti sono comuni, che tramite i prodotti Engim, 
tracciano il percorso delle macchine pulisci neve e spargi sale.
I tracker possono essere prodotti fisici oppure un app per smartphone. A loro 
volta i prodotti fisici si dividono in tracker fissi o mobili. Il servizio include
anche un gestionale per poter visualizzare, modificare o archiviare i propri dati.
\\ Twicetouch è noleggio di dispositivi di sicurezza individuale.
Il prodotto tutela i lavoratori in solitario mandando una segnalazione in caso di
emergenza. Esistono due tipi di rilevazione: 
\begin{itemize}
  \item caduta: l'accelerometro del dispositivo rileva un urto pericoloso
  \item assenza: il dispositivo non si è mosso per un lasso prolungato di tempo, 
  quindi si presume che il lavoratore possa essere incosciente
\end{itemize}
Similmente a servizioGPS è possibile noleggiare un dispositivo fisico (badge) oppure
l'app per android. In entrambi i casi è possibile impostare i numeri in caso di 
emergenza, che riceveranno una chiamata e un messagio SMS.


\subsection{Infrastruttura}
Le tecnologie usate per servizioGPS sono le seguenti:
\begin{itemize}
  \item Ruby on rails full stack
  \item Mariadb e Redis come database
  \item Python come back end di supporto
  \item Java per i prodotti app
\end{itemize}
Il servizio è inoltre diviso su molteplici server per evitare sovraccarico nelle
giornate invernali, quando arrivato migliaia di punti contemporaneamente.
Recentemente è stato un introdotto un server in AWS, aprendo la possibilità di
estrarre logica comune a tutti i server tramite le lambda e quindi rimuovere il
peso dai singoli server.
\subsection{Microservizi}
In alcuni casi non è necessario avere un servizio costantemente acceso che gira,
nello specifico, se devo fare alcune operazioni periodiche o in maniera discontinua
allora non è utile avere un servizio o un api che è sempre in ascolto per un input.
Ecco che entrano in gioco i Microservizi.
\\ A differenza dei servizi, un microservizio non ha un suo server, non ha un suo
ambiente e non dipende da nessun altra tecnologia. Il Microservizio si crea
l'ambiente ogni volta che viene richiamato (cold start), e rimane attivo per un
lasso di tempo, svolge una specifica funzione, definita da un API precisa, e
ritorna un messaggio HTML.
Diventa molto vantaggiosa tale operazione soprattutto nel ambito web dove è molto
naturale usare delle api per fare specifiche operazioni.
\subsection{Sistema automatico di generazione di report}
Le informazioni di ogni cliente vengono archiviate anno dopo anno. Nel caso in cui
il cliente abbia necessità o interesse a salvare i dati del loro lavoro possono
stampare o scaricare una serie di informazioni dal sito. La più importante stampa
sono i punti che arrivano e le loro specifiche. Al momento questo è gestito
internamente tramite una gemma di ruby. Il sistema attuale crea un file HTML,
che viene poi convertito in pdf.
Questo crea una serie di gravi problemi:
\begin{itemize}
  \item Fare il parsing simili è estremamente costoso in maniera computazionale
  \item Per quanto è facile scrivere HTML il controllo sul prodotto finale è basso
  \item La conversione è molto lenta
\end{itemize}
Questa operazione inoltre si presta molto bene ai microservizi. È un operazione
ripetitiva, ben definita che ritorna sempre un file pdf. Quindi sarebbe possibile
evitare calcoli extra di parsing e evitare di sovraccaricare il server se venisse
esportata come microservizio.
Inoltre una volta creata la logica per una singola stampa è possibile allargare
la lamba per la stampa di un qualsiasi altro servizio.



\section{Requisiti del progetto}
\subsection{Descrizione}
Il progetto deve essere una lambda su AWS. La lambda deve essere in grado di
ricevere un json, autenticare tramite token e whitelist, salvare il progetto su
un bucket S3 e ritornare il link di dowload al pdf. Se qualcosa va storto deve
ritornare 400 se è un problema di input oppure 500 se è un problema di connessione
al bucket.
Il programma inoltre deve essere modulare per permettere di espandere in futuro
le stampe ad altri servizi oltre che servizioGPS e creare altri tipi di stampa
come kml e exel.
\subsection{Scelte del linguaggio}
Le lambda su AWS hanno il supporto per i linguaggi più popolari del momento, in
particolare ho preso in considerazione python, javascript e ruby. La libreria
che faceva di più al mio caso era pdfKit di javascript, che è un port di pdfkit
libreria di php per scrivere pdf senza dover far parsing di altri file.
Ruby aveva più libreria con parsing intermedi e python aveva libreria non mantenute.


\section{Sviluppo}
\subsection{Workflow}
Il mio senior ha deciso di fare un implementazione a grossi step, ogni volta
facendo un piccolo pezzo per poi controllare che tutto funzioni correttamente.
Ogni issue è uno step e a ogni step il progetto viene testato sia manualmente su
aws tramite simulazioni di richiesta sia con test automatizzati tramite mocha.

\subsection{Creazione di un file e salvataggio su S3}
La lamda è una funzione assincrona che viene eseguita ogni volta che riceve una
chiamata https. Bisogna prima di tutto creare una connessione con il bucket di s3,
una volta creata questa connessione si può salvare un file passando il filepath
oppure uno stream di data e avremo salvato il file su S3

\subsection{Stampa di un file pdf da un json}
Le operazioni utili al mio scopo text per scrivere una stringa,
stroke per disegnare una linea e infine è possibile manipolare colori e font.
Inizialmente avevo deciso di programmare in maniera funzionale perché risultava
una logica più semplice essendo un breve script, con l'aumentare della complessità
e con il modo di esportare di javascript ho pensato che una classe fosse più adeguata.
Quindi c'è un costruttore che definisce la grandezza del documento, questo mantiene
la relazione di un foglio A4 (circa 1.41) in modo che stampa e visualizzazione sono
sempre molto comodi. Dopo l'inizializzazione si inizia la scrittura dei blocchi.
Ogni json potrebbe avere un header, body e table. Se uno dei 3 manca, non verrà
scritto e gli altri si organizzeranno di conseguenza. Ogni metodo di scrittura
controlla che non ci siano dati mancanti che possano nuocere alla stampa come ad
esempio se manca un titolo non è una stampa valida, e ritornano 400. Se tutto va
bene e la stampa va a buon fine, è possibile mandare l'oggetto con get document.
Ogni possibile caso è testato con mocha, la connessione con AWS è mochata tramite
aws-sdk e gran parte dei dati è generata casualmente tramite faker.
Se tutto va a buon fine il programma ritorna 200 e il link del download.
\end{document}