\documentclass[12pt]{article}
\usepackage{mathptmx}
\usepackage{ragged2e}
\author{}
\title{
    \huge 
        \textbf{Università degli Studi di Modena e Reggio Emilia}
    \large
        \par Dipartimento di Scienze Fisiche, Informatiche e Matematiche
        \par Corso di laurea in Informatica
    \vfil
        \huge \par \textbf{ERS}
    \vfil
}
\date{Anno academico 2018/2019}
\linespread{1.5}
\pagenumbering{arabic}
\begin{document}
\maketitle
\section{Presentazione della situazione esistente}
\subsection{Engim srl}
Engim è una piccola azienda informatica che ha due prodotti principali: servizioGPS e TwiceTouch.
ServizioGPS è un servizio di noleggio di tracker gps ed è il prodotto di punta. 
La maggior parte dei clienti sono comuni che usano i nostri dispositivi per tracciare il 
lavoro delle macchine pulisci neve e spargi sale. 
I tracker sono sia dispositivi fisici da collegare al veicolo che un app per 
android. 
\newline Twicetouch è un servizio di noleggio di dispositivi di sicurezza individuale. 
Il prodotto serve per i lavoratori in solitario, nel evenienza di un qualsiasi 
incidente viene inviata una segnalazione a dei numeri preposti. Questo segnalazione 
avviene nel caso in cui venga avvertito un grosso urto, oppure il lavoratore è 
fermo per un periodo prolungato, presumibilmente svenuto. Simile a servizioGPS c'è 
un prodotto fisico, il badge di TwiceTouch e un app per telefono.

\subsection{Infrastruttura}
Le tecnologie usate per servizioGPS sono le seguenti:
\begin{itemize}
  \item Ruby on rails full stack
  \item Mariadb e Redis come database
  \item Python come back end di supporto
  \item Java per i prodotti app
\end{itemize}
Il servizio è inoltre diviso su molteplici server per evitare sovraccarico nelle 
giornate invernali, quando arrivato migliaia di punti contemporaneamente.
Recentemente è stato un introdotto un server in AWS, aprendo la possibilità di 
estrarre logica comune a tutti i server tramite le lambda e quindi rimuovere il 
peso dai singoli server. 
\subsection{Microservizi}
In alcuni casi non è necessario avere un servizio costantemente acceso che gira, 
nello specifico, se devo fare alcune operazioni periodiche o in maniera discontinua 
allora non è utile avere un servizio o un api che è sempre in ascolto per un input.
Ecco che entrano in gioco i Microservizi. 
\\ A differenza dei servizi, un microservizio non ha un suo server, non ha un suo 
ambiente e non dipende da nessun altra tecnologia. Il Microservizio si crea 
l'ambiente ogni volta che viene richiamato (cold start), e rimane attivo per un 
lasso di tempo, svolge una specifica funzione, definita da un API precisa, e 
ritorna un messaggio HTML. 
Diventa molto vantaggiosa tale operazione soprattutto nel ambito web dove è molto 
naturale usare delle api per fare specifiche operazioni.
\subsection{Sistema automatico di generazione di report}
Le informazioni di ogni cliente vengono archiviate anno dopo anno. Nel caso in cui 
il cliente abbia necessità o interesse a salvare i dati del loro lavoro possono 
stampare o scaricare una serie di informazioni dal sito. La più importante stampa 
sono i punti che arrivano e le loro specifiche. Al momento questo è gestito
internamente tramite una gemma di ruby. Il sistema attuale crea un file HTML, 
che viene poi convertito in pdf. 
Questo crea una serie di gravi problemi:
\begin{itemize}
  \item Fare il parsing simili è estremamente costoso in maniera computazionale
  \item Per quanto è facile scrivere HTML il controllo sul prodotto finale è basso
  \item La conversione è molto lenta 
\end{itemize}
Questa operazione inoltre si presta molto bene ai microservizi. È un operazione 
ripetitiva, ben definita che ritorna sempre un file pdf. Quindi sarebbe possibile 
evitare calcoli extra di parsing e evitare di sovraccaricare il server se venisse 
esportata come microservizio. 
Inoltre una volta creata la logica per una singola stampa è possibile allargare
la lamba per la stampa di un qualsiasi altro servizio. 
\section{Requisiti del progetto}
\subsection{Robe}
% \par Requisiti e progetto
% \par Implementazione
% \par Applicazione di esempio
% \par Conclusioni
% \par Appendici
% \par Bibliografia
\end{document}